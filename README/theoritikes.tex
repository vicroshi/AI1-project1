%Article
\documentclass[14pt,fleqn]{extarticle}
\usepackage[utf8]{inputenc}
\usepackage[main=greek,english]{babel}
\usepackage{amsmath}
\usepackage{graphics}
\usepackage{mathtools}
\usepackage{bm}
\usepackage{xcolor}
\usepackage{changepage,titlesec}
\usepackage[margin=0.5in]{geometry}
\newcommand{\ben}{\begin{otherlanguage}{english}}
\newcommand{\een}{\end{otherlanguage}}
\DeclarePairedDelimiter{\ceil}{\lceil}{\rceil}
\DeclarePairedDelimiter\floor{\lfloor}{\rfloor}
\title{Τεχνητή Νοημοσύνη Ι - Εργασία 1\\[10pt] \underline{\largeΑπαντήσεις θεωρητικών προβλημάτων}}
\author{\LargeΒίκτωρ Λαμπρόπουλος - 1115201900096}
\titleformat{\section}[block]{\bfseries}{\thesection.}{1em}{}
\titleformat{\subsection}[block]{}{\thesubsection}{1em}{}
\titleformat{\subsubsection}[runin]{}{\thesubsubsection}{1em}{}
\titlespacing*{\subsection} {2em}{3.25ex plus 1ex minus .2ex}{1.5ex plus .2ex}
\titlespacing*{\subsubsection} {3em}{3.25ex plus 1ex minus .2ex}{1.5ex plus .2ex}
\begin{document}
\date{}
\maketitle
\begin{center}
\end{center}
\section*{\boxed{\text{Πρόβλημα 2}}}
\begin{adjustwidth}{2em}{0pt}
Με τις προϋπόθεσεις ότι προσθέτουμε τους κόμβους στο σύνορο από αριστερά προς τα δεξιά, ο έλεγχος για κόμβο στόχου γίνεται όταν τον δημιουργούμε και χρησιμοποιούμε στοίβα για την $DLS$ τότε ο μικρότερος αριθμός κόμβων που δημιουργείται θα είναι όταν ο κόμβος στόχος είναι ο πρώτος απ' τα αριστερά του δένδρου.Στο $g-1$ επίπεδο θα έχουμε περάσει από την ρίζα $g$ φορές,από το $1^ο$ επίπεδο $g-1$ κ.ο.κ., ενώ στο τελευταίο επίπεδο θα δημιουργήσουμε μόνο $g+1$. Άρα ο τύπος των διαφανειών γίνεται:\\[8pt]
\[g+(g-1)b+(g-2)b^2+\dots+2b^{g-2} + b^{g-1} + g + 1 = \sum_{i=0}^{g-1}\bigl((g-i)b^i\bigr)+g+1\]

Ο μεγαλύτερος αριθμός αντίστοιχα θα είναι όταν ο κόμβος στόχος βρίσκεται τέρμα δεξιά στο δένδρο και άρα δεν χρειάζεται να αλλάξουμε τον τύπο των διαφανειών.
\[(g+1)+gb+(g-1)b^2+\dots+2b^{g-1}+b^g = \sum_{i=0}^{g}\bigl((g+1-i)b^i\bigr)\]\\
\end{adjustwidth}
\newpage
\section*{\boxed{\text{Πρόβλημα 3}}}
\begin{adjustwidth}{2em}{0pt}
Για τους αλγόριθμους χρησιμοποιώ το $GraphSearch$ της εργασίας του $Pacman$, με την προσθήκη, να μην βάζει στο σύνορο ένα ήδη $expanded$ κόμβο. Το σύνορο έχει κάθε φορά την κατάλληλη δομή για κάθε αναζήτηση, συγκεκριμένα για την $BFS$ μια ουρά, για τις $DFS$,$IDS$ μια στοίβα και για τις $Α^*$,$BestFS$ μία ουρά προτεραιότητας\\[-10pt]
\ben
\begin{verbatim}
    Algorithm: GRAPH_SEARCH:
    frontier = {startNode}
    expanded = {}
    while frontier is not empty:
        node = frontier.pop()
        if isGoal(node):
            return path_to_node
        if node not in expanded:
            expanded.add(node)
            for each child of node's children:
                if child not in expanded:
                    frontier.push(child)
    return failed
\end{verbatim}\een
\begin{equation*}
\begin{aligned}
\\[6pt]&BFS: G_1\\
&[S,A,B,D,G_1]\\[10pt]
&DFS: G_3\\
&[S,D,E,G_3]\\[10pt]
&IDS: G_3\\
&[S] \rightarrow d=0\\&[S,D,B,A] \rightarrow d=1\\
&[S,D,E,C,B,A] \rightarrow d=2\\&[S,D,E,G_3] \rightarrow d=3\\
&\parbox{6.2in}{Εδώ, αν προσθέταμε στο σύνορο και ήδη $expanded$ κόμβους θα βρίσκαμε το $G_1$ για $d=2$}\\[10pt]
\end{aligned}
\end{equation*}
\end{adjustwidth}
\newpage
\section*{\boxed{\text{Πρόβλημα 4}}}
\begin{adjustwidth}{2em}{0pt}
Θεωρούμε ως στοίβα $[a,b,c..]$, όπου τα στοιχεία μπαίνουν και βγαίνουν από την στοίβα στην αρχή της. Επίσης αντιστοιχούμε κάθε αριθμό στο μέγεθος της πίτας επομένως θέλουμε η στοίβα να έχει την μορφή $[a_0,a_1,...a_{n-1}]$ με $a_{0}\leq a_{1}\dots\leq a_{n-1}\leq a_n$
\subsection*{$I$.}
\begin{adjustwidth}{3em}{0pt}
\begin{itemize}
\itemΓια $n=1$, $f(1) = 0$\\
\itemΓια $n=2$, $f(2) = 1$, δηλαδή στην περίπτωση [1,2] θέλουμε 1 φλιπ στην θέση 0\\
\itemΓια $n=3$, $f(3) = 3$, δηλαδή στην χείριστη περίπτωση [1,3,2] θέλουμε 1 φλιπ στην θέση 2 για να φέρουμε την μικρότερη πίτα στην αρχή της στοίβας $\rightarrow [2,3,1]$, μετά 1 φλιπ στην θέση 1 για να έρθει η αμέσως μεγαλύτερη πίτα στην σωστή θέση $\rightarrow [3,2,1]$ και τελευταία 1 φλιπ στο τέλος της στόιβας για να έρθουν όλες στην σωστή θέση  $\rightarrow
[1,2,3]$\\
\itemΓια $n=4$, $f(4) = 4$, ας πάρουμε μία από τις χείριστες περιπτώσεις π.χ. $[3,1,2,4]$, βρίσκουμε την μικρότερη πίτα στην θέση 1 και κάνουμε 1 φλιπ $\rightarrow[3,4,2,1]$, μετά πάμε στην αρχή της στοίβας και κάνουμε 1 φλιπ $\rightarrow[1,2,4,3]$, 1 φλιπ στην θέση 2 $\rightarrow[1,2,3,4]$ και τέλος 1 φλιπ στην αρχή $\rightarrow[4,3,2,1]$
\end{itemize}
\end{adjustwidth}
\subsection*{$II$.}
\begin{adjustwidth}{3em}{0pt}
Αν σκεφτούμε την στοίβα ως μια ακολουθία διαδοχικών ζευγαριών, τότε τα ζευγάρια των οποίων η απόλυτη διαφορά είναι μεγαλύτερη του 1 είναι λάθος και θα πρέπει να χωριστούν, αυτό γίνεται μόνο άμα βάλουμε την σπάτουλα ανάμεσα τους, άρα για κάθε τέτοιο ζευγάρι που προκύπτει η υπάρχει εξ αρχής θα χρειαστεί ένα φλίπ. Αν τώρα $n\geq4$ και είναι άρτιος, τότε στην χείριστη περίπτωση μπορούμε να έχουμε $n-1$ λανθασμένα ζευγάρια και άμα συμπεριλάβουμε το ζευγάρι,δηλαδή τελευταία πίτα της στοίβας με πιάτο (τέλος στοίβας), θα έχουμε στην χειρότερη $n$. Αυτή είναι όταν όλες οι μονές πίτες $\leq n-1$ (δηλ, η 1η μικρότερη, η 3η μικρότερη κ.ο.κ) μπουν πρώτες πάνω από το πιάτο, με την προϋπόθεση πως η μεγαλύτερη απο αυτές δεν πρέπει να μπει τελευταία $n$-οστή πίτα,έπειτα θα μπουν όλες οι άρτιες $\leq n$ (δηλ, η 2η μικρότερη, η 4η μικρότερη κ.ο.κ) με την υπόθεση ότι η μικρότερη(2η πίτα) δεν μπορεί να προηγηθεί των άλλων. Ένα παράδειγμα είναι το εξής:\\[7pt]
$[2,4,6,8...,n,1,3,5,9...,n-1]$\\[7pt]Έτσι θα χρειαστούν τουλάχιστον $n$ φλιπ, και άρα  $f(n)\geq n,  \forall n\geq4$, όπου  $n$ άρτιος.
\newpage
Αν τώρα $n$ περιττός, τότε αλλάζει η σειρά, δηλάδη πρώτα μπαίνουν οι άρτιες πίτες με την μεγαλύτερη να μην μπορεί να μπει τελευταία και έπειτα οι μονές με την μικρότερη να μην μπορεί να μπει πρώτη. Το προγούμενο παράδειγμα γίνεται λοιπόν:\\[7pt]
$[1,3,5,9..,n,2,4,6,8...n-1]$\\[7pt]Πάλι έχουμε $n$  ζεύγη και άρα $f(n)\geqn,\foralln\geq4$, όπου $n$ περιττός.\\
Στις παραπάνω περιπτώσεις έχουμε $n>=4$ αφού εύκολα βλέπουμε ότι δεν μπορούμε να σχηματίσουμ $n$ λανθασμένα ζευγάρια.
\end{adjustwidth}
\subsection*{$III$.}
\begin{adjustwidth}{3em}{0pt}
\end{adjustwidth}
\subsection*{$IV$.}
Για κάθε στοίβα με πίτες μεγέθους $n$ έχουμε $n!$ διαφορετικούς συνδιασμούς. Μπορούμε να το σκεφτούμε σαν έναν μη-κατευθυνόμενο γράφο με $n!$ κορυφές και με κάθε μία απο αυτές να έχει $n-1$ ακμές, κάθε ακμή συμβολίζει ένα φλιπ. Ξεκινώντας από έναν κόμβο του γράφου (κατάσταση) , ο στόχος μας είναι να φτάσουμε στη σωστά ταξινομημένη κορυφή. Το μέγεθους του χώρου αναζήτης είναι $n!$
\end{adjustwidth}
\end{document}
